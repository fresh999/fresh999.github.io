\documentclass[a4paper, 12pt]{article}

\usepackage{amsmath, amsthm, amssymb}
\usepackage{titlesec}

\usepackage{hyperref}
\usepackage[capitalize]{cleveref}
\hypersetup{
    colorlinks=true,
    allcolors=black,
}

\theoremstyle{definition}
\newtheorem{definition}{Definition}[section]

\theoremstyle{plain}
\newtheorem{theorem}[definition]{Theorem}
\newtheorem{proposition}[definition]{Proposition}
\newtheorem{corollary}[definition]{Corollary}
\newtheorem{lemma}[definition]{Lemma}

\theoremstyle{remark}
\newtheorem{remark}[definition]{Remark}
\newtheorem{example}[definition]{Example}

\renewcommand\qedsymbol{$\square$}
\renewcommand{\theenumi}{(\roman{enumi})}
\renewcommand{\labelenumi}{(\roman{enumi})}

\let\existstemp\exists
\let\foralltemp\forall
\renewcommand{\exists}{\mkern18mu\existstemp\mkern2mu}
\renewcommand{\forall}{\mkern18mu\foralltemp\mkern2mu}

\newcommand\isoto{\stackrel{\sim}{\smash{\longrightarrow}\rule{0pt}{0.4ex}}}

\usepackage[toc]{appendix}
\usepackage[backend=biber, style=numeric]{biblatex}
\usepackage[english]{babel}
\usepackage{csquotes}
\usepackage[inline]{enumitem}
\usepackage[margin=1in]{geometry}
\usepackage{mathtools}
\usepackage{physics}
%\usepackage{subfiles}
\usepackage{tikz-cd}

\begin{document}

Today we take a quick look at the definition of a Lie group. A Lie group is a group that is simultaneously a differentiable manifold and such that the group operations are smooth. An example of a Lie group is \( \operatorname{GL}_n \mathbb{R} \), that is the set of all \( n \times n \) real matrices, together with the Euclidean topology (\( \operatorname{GL}_n \mathbb{R} \sim \mathbb{R}^{n^2} \))

\begin{proof}
    This is a not a proof.
\end{proof}

Hey

\begin{equation}
    n = 2
\end{equation}

\begin{theorem}
    Hey
\end{theorem}

\begin{lemma}
    Hey
\end{lemma}

hòsikjdvbksljvb

\begin{proposition}
    Hey hey
    \begin{equation}
        n = 2056
    \end{equation}
\end{proposition}

\begin{enumerate}
    \item First item
    \item second item
    \item third item
\end{enumerate}

\end{document}
