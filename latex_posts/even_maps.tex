\documentclass[a4paper, 12pt]{article}

\usepackage{amsmath, amsthm, amssymb}
\usepackage{titlesec}

\usepackage{hyperref}
\usepackage[capitalize]{cleveref}
\hypersetup{
    colorlinks=true,
    allcolors=black,
}

\theoremstyle{definition}
\newtheorem{definition}{Definition}[section]

\theoremstyle{plain}
\newtheorem{theorem}[definition]{Theorem}
\newtheorem{proposition}[definition]{Proposition}
\newtheorem{corollary}[definition]{Corollary}
\newtheorem{lemma}[definition]{Lemma}

\theoremstyle{remark}
\newtheorem{remark}[definition]{Remark}
\newtheorem{example}[definition]{Example}

\renewcommand\qedsymbol{$\square$}
\renewcommand{\theenumi}{(\roman{enumi})}
\renewcommand{\labelenumi}{(\roman{enumi})}

\let\existstemp\exists
\let\foralltemp\forall
\renewcommand{\exists}{\mkern18mu\existstemp\mkern2mu}
\renewcommand{\forall}{\mkern18mu\foralltemp\mkern2mu}

\newcommand\isoto{\stackrel{\sim}{\smash{\longrightarrow}\rule{0pt}{0.4ex}}}

\usepackage[toc]{appendix}
\usepackage[backend=biber, style=numeric]{biblatex}
\usepackage[english]{babel}
\usepackage{csquotes}
\usepackage[inline]{enumitem}
\usepackage[margin=1in]{geometry}
\usepackage{mathtools}
\usepackage{physics}
%\usepackage{subfiles}
\usepackage{tikz-cd}

\begin{document}

Hi everyone and welcome back to a brand new post! Today discuss even maps on spheres and their degree. Specifically, we show that they always had even degree.

Recall that a map \( f \colon S^n \to S^n \) is called \textit{even} if \( f(x) = f(-x) \) for all \( x \in S^n \). Its degree is defined as the unique integer \( m \) such that the induced map on homology

\begin{equation*}
    f_* \colon H_n(S^n) \to H_n(S^n)
\end{equation*}

is multiplication by \( m \).

\begin{proposition}
    An even map \( f \) has even degree. Moreover, if \( n \) is even, \( \deg f = 0 \).
\end{proposition}

\begin{proof}
    Let us first suppose \( n \) is even. Denoting by \( a \) the antipodal map on \( S^n \),

    \begin{equation*}
        \deg f = \deg f \circ a = (-1)^{n+1} \deg f = - \deg f
    \end{equation*}

    Thus, \( \deg f = 0 \).

    Let now \( n \) be odd. Since \( f \) is even, it factors through \( \mathbb{R}P^n \):

    \begin{equation*}
        S^n \xrightarrow{\pi} \mathbb{R}P^n \xrightarrow{\widetilde{f}} S^n
    \end{equation*}

    By functoriality of \( H_n \), it suffices to show that the induced map \( \pi_* \) takes a generator of \( H_n(S^n) \) to twice a generator of \( H_n(\mathbb{R}P^n) \).
    To do this, we look at \( S^n \) as a cell complex with \( 2 \) \( i \)-cells for \( i = 0, 1, \dots, n \), namely the two hemipheres glued along the equatorial. Thus, the celular chain complex of \( S^n \) is

    \begin{equation*}
        0 \longrightarrow \mathbb{Z}^2 \longrightarrow \dots \longrightarrow \mathbb{Z}^2 \longrightarrow 0
    \end{equation*}

    Let \( e_1^i, e_2^i \) be the two \( i \)-cells. Using the cellular boundary formula, we see that
    \begin{equation*}
        \partial e_k^i = e_1^{i-1} + (-1)^{i}e_2^{i-1}.
    \end{equation*}

    Thus, \( H_n(S^n) \) is generated by the homology class \( \langle e^n_1 - e_2^n \rangle \) and, denoting the only \( n \)-cell of \( \mathbb{R}P^n \) by \( e^n \), we have

    \begin{equation*}
        \pi_*(\langle e^n_1 - e_2^n \rangle) = 2 \langle e^n \rangle.
    \end{equation*}

    Again by the cellular boundary formula, we see that \( \langle e^n \rangle \) generates \( H_n(\mathbb{R}P^n) \). This completes the proof.
\end{proof}

Until next time folks, stay fresh!

\end{document}
