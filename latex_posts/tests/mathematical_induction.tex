\documentclass[a4paper, 12pt]{article}

\usepackage{amsmath, amsthm, amssymb}
\usepackage{titlesec}

\usepackage{hyperref}
\usepackage[capitalize]{cleveref}
\hypersetup{
    colorlinks=true,
    allcolors=black,
}

\theoremstyle{definition}
\newtheorem{definition}{Definition}[section]

\theoremstyle{plain}
\newtheorem{theorem}[definition]{Theorem}
\newtheorem{proposition}[definition]{Proposition}
\newtheorem{corollary}[definition]{Corollary}
\newtheorem{lemma}[definition]{Lemma}

\theoremstyle{remark}
\newtheorem{remark}[definition]{Remark}
\newtheorem{example}[definition]{Example}

\renewcommand\qedsymbol{$\square$}
\renewcommand{\theenumi}{(\roman{enumi})}
\renewcommand{\labelenumi}{(\roman{enumi})}

\let\existstemp\exists
\let\foralltemp\forall
\renewcommand{\exists}{\mkern18mu\existstemp\mkern2mu}
\renewcommand{\forall}{\mkern18mu\foralltemp\mkern2mu}

\newcommand\isoto{\stackrel{\sim}{\smash{\longrightarrow}\rule{0pt}{0.4ex}}}

\usepackage[toc]{appendix}
%\usepackage[backend=biber, style=numeric]{biblatex}
\usepackage[english]{babel}
\usepackage{csquotes}
\usepackage[inline]{enumitem}
\usepackage[margin=1in]{geometry}
\usepackage{mathtools}
\usepackage{physics}
%\usepackage{subfiles}
\usepackage{tikz-cd}

\begin{document}

In this post, we explain the concept of mathematical induction through a well-known example, that is the sum of the first \( n \) natural numbers.

Mathematical induction is very useful and is employed quite frequently in top-level mathematics.

Let us consider the sum
\begin{equation}
    \sum_{j = 1}^n j
\end{equation}

We want to prove that it is equal to \( \frac{n(n+1)}{2} \) for all \( n \in \mathbb{N} \). It is trivial to see that the equality holds in the case \( n = 1 \). Suppose now now that it holds for \( n \in \mathbb{N} \) and let us prove that it also holds \( n + 1 \).

\begin{align*}
    \sum_{j = 1}^{n+1} j &= \sum_{j = 1}^n j + n+1 \\
    &= \frac{n(n+1)}{2} + (n+1)\\
    &= \frac{(n+1)(n+2)}{2}
\end{align*}
\begin{equation}
    n = 1
\end{equation}
\begin{equation}
    n = 2
\end{equation}

\end{document}
